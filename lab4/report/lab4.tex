\documentclass[a4paper,14pt]{extarticle}

\usepackage[utf8x]{inputenc}
\usepackage[T1]{fontenc}
\usepackage[russian]{babel}
\usepackage{hyperref}
\usepackage{indentfirst}
\usepackage{here}
\usepackage{array}
\usepackage{graphicx}
\usepackage{caption}
\usepackage{subcaption}
\usepackage{chngcntr}
\usepackage{amsmath}
\usepackage{amssymb}
\usepackage[left=2cm,right=2cm,top=2cm,bottom=2cm,bindingoffset=0cm]{geometry}
\usepackage{multicol}
\usepackage{multirow}
\usepackage{titlesec}
\usepackage{listings}
\usepackage{color}
\usepackage{enumitem}
\usepackage{cmap}
\usepackage{underscore}

\definecolor{green}{rgb}{0,0.6,0}
\definecolor{gray}{rgb}{0.5,0.5,0.5}
\definecolor{purple}{rgb}{0.58,0,0.82}

\lstdefinelanguage{none}{}

\lstset{
	language={Java},
	inputpath={../generator/src/main/java/com/vaddya/hotelbooking},
	backgroundcolor=\color{white},
	commentstyle=\color{green},
	keywordstyle=\color{blue},
	numberstyle=\scriptsize\color{gray},
	stringstyle=\color{purple},
	basicstyle=\ttfamily\small,
	breakatwhitespace=false,
	breaklines=true,
	captionpos=b,
	keepspaces=true,
	numbers=left,
	numbersep=5pt,
	showspaces=false,
	showstringspaces=false,
	showtabs=false,
	tabsize=4,
	texcl=true,
	extendedchars=false,
	frame=single,
	morekeywords={IF, BIGSERIAL, SERIAL, TEXT, BIGINT, MONEY, BOOLEAN, REFERENCES}
}

\renewcommand{\le}{\ensuremath{\leqslant}}
\renewcommand{\leq}{\ensuremath{\leqslant}}
\renewcommand{\ge}{\ensuremath{\geqslant}}
\renewcommand{\geq}{\ensuremath{\geqslant}}
\renewcommand{\epsilon}{\ensuremath{\varepsilon}}
\renewcommand{\phi}{\ensuremath{\varphi}}
\renewcommand{\thefigure}{\arabic{figure}}
\newcommand{\code}[1]{\texttt{#1}}
\newcommand{\caret}{\^{}}

\titleformat*{\section}{\large\bfseries} 
\titleformat*{\subsection}{\normalsize\bfseries} 
\titleformat*{\subsubsection}{\normalsize\bfseries} 
\titleformat*{\paragraph}{\normalsize\bfseries} 
\titleformat*{\subparagraph}{\normalsize\bfseries} 

\counterwithin{figure}{section}
\counterwithin{equation}{section}
\counterwithin{table}{section}
\newcommand{\sign}[1][5cm]{\makebox[#1]{\hrulefill}}
\newcommand{\equipollence}{\quad\Leftrightarrow\quad}
\newcommand{\no}[1]{\overline{#1}}
\graphicspath{{../pics/}}
\captionsetup{justification=centering,margin=1cm}
\def\arraystretch{1.3}
\setlength\parindent{5ex}
\titlelabel{\thetitle.\quad}

\setitemize{topsep=0.5em, itemsep=0em}
\setenumerate{topsep=0.5em, itemsep=0em}

\begin{document}

\begin{titlepage}
\begin{center}
	Санкт-Петербургский Политехнический Университет Петра Великого\\[0.3cm]
	Институт компьютерных наук и технологий \\[0.3cm]
	Кафедра компьютерных систем и программных технологий\\[4cm]
	
	\textbf{ОТЧЕТ}\\ 
	\textbf{по лабораторной работе}\\[0.5cm]
	\textbf{<<Разработка структуры базы данных>>}\\[0.1cm]
	Базы данных\\[3.0cm]
\end{center}

\begin{flushright}
	\begin{minipage}{0.45\textwidth}
		\textbf{Работу выполнил студент}\\[3mm]
		группа 43501/3 \hfill Дьячков В.В.\\[5mm]
		\textbf{Работу принял преподаватель}\\[5mm]
		\sign[3cm] \hfill Мяснов А.В. \\[5mm]
	\end{minipage}
\end{flushright}

\vfill

\begin{center}
	Санкт-Петербург\\[0.3cm]
	\the\year
\end{center}
\end{titlepage}

\addtocounter{page}{1}

\tableofcontents
\newpage

\section{Цель работы}

Познакомиться с языком создания запросов управления данными SQL-DML.

\section{Программа работы}

\begin{enumerate}
	\item Изучение SQL-DML.
	\item Выполнение всех запросов из списка стандартных запросов. Демонстрация результатов преподавателю.
	\item Получение у преподавателя и реализация SQL-запросов в соответствии с индивидуальным заданием. Демонстрация результатов преподавателю.
	\item Сохранение в БД выполненных запросов SELECT в виде представлений, запросов INSERT, UPDATE или DELETE -- в виде ХП. Выкладывание скрипта в GitLab.
\end{enumerate}

\section{Теоретическая информация}

\section{Стандартные запросы}

\subsection{Выборка всех данных}

Сделайте выборку всех данных из каждой таблицы

\subsection{Выборка с использованием логических операций}

Сделайте выборку данных из одной таблицы при нескольких условиях, с использованием логических операций, \code{LIKE}, \code{BETWEEN}, \code{IN} (не менее 3-х разных примеров)

\subsection{Запрос с вычисляемым полем}

Создайте в запросе вычисляемое поле

\subsection{Выборка с использованием сортировки}

Сделайте выборку всех данных с сортировкой по нескольким полям

\subsection{Запрос с вычислением совокупных характеристик таблиц}

Создайте запрос, вычисляющий несколько совокупных характеристик таблиц

\subsection{Выборка из связанных таблиц}

Сделайте выборку данных из связанных таблиц (не менее двух примеров)

\subsection{Запрос с использованием группировки}

Создайте запрос, рассчитывающий совокупную характеристику с использованием группировки, наложите ограничение на результат группировки

\subsection{Вложенный запрос}

Придумайте и реализуйте пример использования вложенного запроса

\subsection{Вставка записей}

С помощью оператора \code{INSERT} добавьте в каждую таблицу по одной записи

\subsection{Изменение записей}

С помощью оператора \code{UPDATE} измените значения нескольких полей у всех записей, отвечающих заданному условию

\subsection{Удаление записей по условию}

С помощью оператора \code{DELETE} удалите запись, имеющую максимальное (минимальное) значение некоторой совокупной характеристики

\subsection{Удаление с использованием вложенного запроса}

С помощью оператора \code{DELETE} удалите записи в главной таблице, на которые не ссылается подчиненная таблица (используя вложенный запрос)

\section{Запросы в соответствие с задание преподавателя}

\subsection{Рейтинг городов по кварталам}

\paragraph{Задание:} Вывести рейтинг городов по кварталам. В рейтинге 5 городов в, которые больше всего ездят в каком-то квартале.

\lstinputlisting[caption=\code{quarter.sql}]{quarter.sql}

\subsection{Клиенты, имеющие наибольший средний рост стоимости путевки}

\paragraph{Задание:} Вывести 5 клиентов, которые имеют наибольший средний рост стоимости путевки.

\lstinputlisting[caption=\code{travellers.sql}]{travellers.sql}

\section{Выводы}



\end{document}
