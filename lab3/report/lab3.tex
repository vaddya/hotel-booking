\documentclass[a4paper,14pt]{extarticle}

\usepackage[utf8x]{inputenc}
\usepackage[T1]{fontenc}
\usepackage[russian]{babel}
\usepackage{hyperref}
\usepackage{indentfirst}
\usepackage{here}
\usepackage{array}
\usepackage{graphicx}
\usepackage{caption}
\usepackage{subcaption}
\usepackage{chngcntr}
\usepackage{amsmath}
\usepackage{amssymb}
\usepackage[left=2cm,right=2cm,top=2cm,bottom=2cm,bindingoffset=0cm]{geometry}
\usepackage{multicol}
\usepackage{multirow}
\usepackage{titlesec}
\usepackage{listings}
\usepackage{color}
\usepackage{enumitem}
\usepackage{cmap}
\usepackage{underscore}

\definecolor{green}{rgb}{0,0.6,0}
\definecolor{gray}{rgb}{0.5,0.5,0.5}
\definecolor{purple}{rgb}{0.58,0,0.82}

\lstdefinelanguage{none}{}

\lstset{
	language={Java},
	inputpath={../generator/src/main/java/com/vaddya/hotelbooking},
	backgroundcolor=\color{white},
	commentstyle=\color{green},
	keywordstyle=\color{blue},
	numberstyle=\scriptsize\color{gray},
	stringstyle=\color{purple},
	basicstyle=\ttfamily\small,
	breakatwhitespace=false,
	breaklines=true,
	captionpos=b,
	keepspaces=true,
	numbers=left,
	numbersep=5pt,
	showspaces=false,
	showstringspaces=false,
	showtabs=false,
	tabsize=4,
	texcl=true,
	extendedchars=false,
	frame=single,
	morekeywords={IF, BIGSERIAL, SERIAL, TEXT, BIGINT, MONEY, BOOLEAN, REFERENCES}
}

\renewcommand{\le}{\ensuremath{\leqslant}}
\renewcommand{\leq}{\ensuremath{\leqslant}}
\renewcommand{\ge}{\ensuremath{\geqslant}}
\renewcommand{\geq}{\ensuremath{\geqslant}}
\renewcommand{\epsilon}{\ensuremath{\varepsilon}}
\renewcommand{\phi}{\ensuremath{\varphi}}
\renewcommand{\thefigure}{\arabic{figure}}
\newcommand{\code}[1]{\texttt{#1}}
\newcommand{\caret}{\^{}}

\titleformat*{\section}{\large\bfseries} 
\titleformat*{\subsection}{\normalsize\bfseries} 
\titleformat*{\subsubsection}{\normalsize\bfseries} 
\titleformat*{\paragraph}{\normalsize\bfseries} 
\titleformat*{\subparagraph}{\normalsize\bfseries} 

\counterwithin{figure}{section}
\counterwithin{equation}{section}
\counterwithin{table}{section}
\newcommand{\sign}[1][5cm]{\makebox[#1]{\hrulefill}}
\newcommand{\equipollence}{\quad\Leftrightarrow\quad}
\newcommand{\no}[1]{\overline{#1}}
\graphicspath{{../pics/}}
\captionsetup{justification=centering,margin=1cm}
\def\arraystretch{1.3}
\setlength\parindent{5ex}
\titlelabel{\thetitle.\quad}

\setitemize{topsep=0.5em, itemsep=0em}
\setenumerate{topsep=0.5em, itemsep=0em}

\begin{document}

\begin{titlepage}
\begin{center}
	Санкт-Петербургский Политехнический Университет Петра Великого\\[0.3cm]
	Институт компьютерных наук и технологий \\[0.3cm]
	Кафедра компьютерных систем и программных технологий\\[4cm]
	
	\textbf{ОТЧЕТ}\\ 
	\textbf{по лабораторной работе}\\[0.5cm]
	\textbf{<<Разработка структуры базы данных>>}\\[0.1cm]
	Базы данных\\[3.0cm]
\end{center}

\begin{flushright}
	\begin{minipage}{0.45\textwidth}
		\textbf{Работу выполнил студент}\\[3mm]
		группа 43501/3 \hfill Дьячков В.В.\\[5mm]
		\textbf{Работу принял преподаватель}\\[5mm]
		\sign[3cm] \hfill Мяснов А.В. \\[5mm]
	\end{minipage}
\end{flushright}

\vfill

\begin{center}
	Санкт-Петербург\\[0.3cm]
	\the\year
\end{center}
\end{titlepage}

\addtocounter{page}{1}

\tableofcontents
\newpage

\section{Цель работы}

Сформировать набор данных, позволяющий производить операции на реальных объемах данных.

\section{Программа работы}

\begin{enumerate}
	\item Реализация в виде программы параметризуемого генератора, который позволит сформировать набор связанных данных в каждой таблице.
	\item Частные требования к генератору, набору данных и результирующему набору данных:
		\begin{itemize}[topsep=0em]
			\item количество записей в справочных таблицах должно соответствовать ограничениям предметной области;
			\item количество записей в таблицах, хранящих информацию об объектах или субъектах должно быть параметром генерации;
			\item значения для внешних ключей брать из связанных таблиц.
		\end{itemize}
\end{enumerate}

\section{Выполнение работы}

\subsection{Выбор технологий для реализация генератора}

Для реализации генератора был использован язык программирования \textbf{Java}. 

Для взаимодействия Java с базой данных используется технология \textbf{JDBC} (Java Database Connectivity), которая обеспечивает доступ Java API к реляционным базам данных (обеспечивает соединение с БД) и позволяет создавать SQL-выражения, выполнять SQL-запросы, просматривать и модифицировать записи. 

Для доступа к каждой конкретной БД необходим специальный JDBC-драйвер, который является адаптером Java-приложения к БД. Например, для доступа к базе данных PostgreSQL используется драйвер, который предоставляет PostgreSQL. 

Минусом использования JDBC напрямую является необходимость написания большого количества однообразного кода и ручного формирования SQL-запросов для выборки и изменения данных. Вместо этого была выбрана технология \textbf{ORM} (Object-Relational Mapping), которая связывает базы данных с концепциями объектно-ориентированных языков, создавая <<виртуальную объектную базу данных>>. Самым распространенным ORM-фреймворком для Java является \textbf{Hibernate}.

ORM является еще одним уровнем абстракции, построенным поверх JDBC. Hibernate позволяет отобразить сущности, хранящиеся в РБД, на Java-классы, упростив при этом выборку и обновление данных. С помощью конфигурационных файлов и Java-аннотаций можно указать Hibernate как извлекать данные из класса и соединять с определенным столбцами в таблице БД. Кроме этого в конфигурационном файле указывается диалект базы данных (в нашем случае \code{org.hibernate.dialect.PostgreSQL9Dialect}), класс JDBC драйвера (\code{org.postgresql.Driver}), URL, имя пользователя и пароль к базе данных и др.

В Hibernate для получения физического соединения с базой данных  используется сессия (session). Благодаря тому, что сессия является легковесны объектом. С помощью сессий появляется возможность создавать, читать, редактировать и удалять объекты внутри базы данных.

\subsection{Реализация сущностей прикладной области}

Для работы с Hibernate необходимо создать Persistent Classes, чтобы отобразить на них сущности из реляционной базы данных. Такие классы удобно реализовать в виде POJO (Plain Old Java Object) классов, содержащих пустой конструктор, private поля, соответствующие полям соответствующий таблицы РБД, и get/set методы для них. Помимо этого внутри данных классов используются аннотации из пакета \code{javax.persistence}:
\begin{itemize}
	\item \code{@Entity} для пометки POJO класса как JPA (Java Persistence API) сущности;
	\item \code{@Table} для указание имени соответствующей таблицы базы данных;
	\item \code{@Id} для пометки первичного ключа таблицы;
	\item \code{@GeneratedValue} используется совестно с \code{@Id} и указывает на то, что ключ генерируется базой данных;
	\item \code{@Column} для указания имени, длины, nullability и уникальности соответствующего столбца таблицы;
	\item \code{@OneToOne}, \code{@OneToMany}, \code{@ManyToOne} и \code{@ManyToMany} для указания соответствующих связей между сущностями и др.
\end{itemize}

Все POJO классы прикладной области (бронирование отелей), такие как \code{User}, \code{Hotel}, \code{Reservation} и др. были реализованы на языке Java внутри пакета \code{com.vaddya.hotelbooking.model}.

\subsection{Реализация сущностей доступа к данным}

Для описания доступа к данным был разработан параметризованный интерфейс \code{Dao<E, I extends Serializable>}, в котором указаны методы, которые необходимо реализовать для манипулирования некоторой сущностью прикладной области \code{E}, обладающей ключом с типом \code{I}.

\lstinputlisting[caption=\code{Dao.java}]{dao/Dao.java}

Общая для всех сущностей логика была реализована в абстрактном классе \code{EntityDao}, внутри которого используется объект сессии для получения и обновления данных с использованием транзакций. Для всех сущностей прикладной области внутри пакета \code{com.vaddya.hotelbooking.dao} были созданы DAO (Data Access Object) классы, являющиеся наследниками \code{EntityDao}. С их помощью прикладная программа (генератор), взаимодействует с базой данных.

\subsection{Реализация генератора тестовых данных}

Для генерации тестовых данных была использована библиотека \textbf{Faker}, которая является портированной на Java библиотекой для языка Ruby. Faker позволяет генерировать осмысленные значения для полей имени, адреса, телефона и др. Логика создания случайных тестовых объектов для каждой сущности прикладной области была реализована в классе \code{EntityGenerator} в пакете \code{com.vaddya.hotelbooking.generator}. В том же пакете был реализован класс \code{HibernateSessionFactory} для настройки фабрики Hibernate сессий. 

В классах \code{GeneratorOptions} и \code{Generator} реализована основная логика работы генератора. Внутри \code{GeneratorOptions} используется библиотека Apache Commons CLI для удобного разбора аргументов командой строки генератора и формирования поясняющего вывода. Генерирование данных может происходить в нескольких режимах в зависимости от параметров командой строки:
\begin{itemize}
	\item \code{-a,--all} -- генерация всех сущностей;
	\item \code{-c,--cluster} -- <<кластерная>> генерация данных:
		\begin{enumerate}
			\item \code{city} -- генерация только стран, городов и пользователей;
			\item \code{hotel} -- генерация только отелей и их правил, тип комнат и удобств в них, комнат и цен;
			\item \code{reservation} -- генерация только бронирований и бонусов/штрафов, гостей, отмен и отзывов к ним.
		\end{enumerate}
\end{itemize}

Для каждого из способов можно указать только \code{-n,--number} -- базовое число, относительно котрого будет автоматически выбрано соответствующее число элементов той или иной сущности. При этом если не указать его или число генерируемых данных, то будет использовано значение по умолчанию.

\begin{lstlisting}[language=none]
usage: generator [-?] [-a] [-b <arg>] [-c <arg>] [-f <arg>] [-g] [-h <arg>] 
[-i <arg>] [-l <arg>] [--max-bp <arg>] [--max-facilities <arg>] 
[--min-bp <arg>] [--min-facilities <arg>] [-n <arg>] [-o <arg>] 
[-p] [-r <arg>] [-s <arg>] [-t] [-u <arg>] [-v <arg>] [-w <arg>]
	-?,--help                    print help
	-a,--all                     generate all entities
	-b,--bonus-penalties <arg>   number of bonuses or penalties, default: 20
	-c,--cluster <arg>           cluster type: [city | hotel | reservation]
	-f,--facilities <arg>        number of facilities, default: 100
	-g,--guests                  generate guests
	-h,--hotels <arg>            number of hotels, default: 50
	-i,--cities <arg>            number of cities, default: 100
	-l,--cancellations <arg>     number of cancellations, default: 100
	--max-bp <arg>               maximum number of bonuses or penalties per
		reservation, default: 2
	--max-facilities <arg>       maximum number of facilities per room,
		default 30
	--min-bp <arg>               minimum number of bonuses or penalties per
		reservation, default: 0
	--min-facilities <arg>       minimum number of facilities per room,
		default 10
	-n,--number <arg>            base number of cluster, default 1000
	-o,--counties <arg>          number of countries, default: 10
	-p,--prices                  generate prices for room types
	-r,--rooms <arg>             number of rooms, default: 1000
	-s,--house-rules <arg>       number of house rules, default: 250
	-t,--room-types              generate room types
	-u,--users <arg>             number of users, default: 5000
	-v,--reservations <arg>      number of reservations, default: 50000
	-w,--reviews <arg>           number of reviews, default: 20000
\end{lstlisting}

В классе \code{Generator} производится анализ аргументов командой строки, после чего с помощью \code{EntityGenerator} происходит генерирование нужного количества сущностей каждого типа, которые вставляются в базу данных при помощи DAO классов.

\section{Выводы}

В процессе данной работы был разработан генератор тестовых данных на языке Java с использование ORM-библиотеки Hibernate. ORM подход позволяет избежать ручного написания SQL-запросов при помощи отображения таблиц реляционной базы данных на Java-классы.

\end{document}
