\documentclass[a4paper,14pt]{extarticle}

\usepackage[utf8x]{inputenc}
\usepackage[T1]{fontenc}
\usepackage[russian]{babel}
\usepackage{hyperref}
\usepackage{indentfirst}
\usepackage{here}
\usepackage{array}
\usepackage{graphicx}
\usepackage{caption}
\usepackage{subcaption}
\usepackage{chngcntr}
\usepackage{amsmath}
\usepackage{amssymb}
\usepackage[left=2cm,right=2cm,top=2cm,bottom=2cm,bindingoffset=0cm]{geometry}
\usepackage{multicol}
\usepackage{multirow}
\usepackage{titlesec}
\usepackage{listings}
\usepackage{color}
\usepackage{enumitem}
\usepackage{cmap}
\usepackage{listingsutf8}

\definecolor{green}{rgb}{0,0.6,0}
\definecolor{gray}{rgb}{0.5,0.5,0.5}
\definecolor{purple}{rgb}{0.58,0,0.82}

\lstdefinelanguage{none}{}

\lstset{
	language={SQL},
	inputpath={../sql/},
	backgroundcolor=\color{white},
	commentstyle=\color{green},
	keywordstyle=\color{blue},
	numberstyle=\color{gray}\scriptsize\ttfamily,
	stringstyle=\color{purple},
	basicstyle=\lst@ifdisplaystyle\footnotesize\fi\ttfamily,
	breakatwhitespace=false,
	breaklines=true,
	captionpos=b,
	keepspaces=true,
	numbers=left,
	numbersep=5pt,
	showspaces=false,
	showstringspaces=false,
	showtabs=false,
	tabsize=4,
	frame=single,
	morekeywords={IF, BIGSERIAL, SERIAL, TEXT, BIGINT, MONEY, BOOLEAN, REFERENCES},
	deletekeywords={count, error},
	morecomment=[l][\color{green}]{\$\$},
	sensitive=true,
	columns=fullflexible,
	inputencoding=utf8/cp1251
}

\renewcommand{\le}{\ensuremath{\leqslant}}
\renewcommand{\leq}{\ensuremath{\leqslant}}
\renewcommand{\ge}{\ensuremath{\geqslant}}
\renewcommand{\geq}{\ensuremath{\geqslant}}
\renewcommand{\epsilon}{\ensuremath{\varepsilon}}
\renewcommand{\phi}{\ensuremath{\varphi}}
\renewcommand{\thefigure}{\arabic{figure}}
\newcommand{\code}[1]{\lstinline|#1|}
\newcommand{\caret}{\^{}}

\titleformat*{\section}{\large\bfseries} 
\titleformat*{\subsection}{\normalsize\bfseries} 
\titleformat*{\subsubsection}{\normalsize\bfseries} 
\titleformat*{\paragraph}{\normalsize\bfseries} 
\titleformat*{\subparagraph}{\normalsize\bfseries} 

\counterwithin{figure}{section}
\counterwithin{equation}{section}
\counterwithin{table}{section}
\newcommand{\sign}[1][5cm]{\makebox[#1]{\hrulefill}}
\newcommand{\equipollence}{\quad\Leftrightarrow\quad}
\newcommand{\no}[1]{\overline{#1}}
\graphicspath{{../../migrations/}}
\captionsetup{justification=centering,margin=1cm}
\def\arraystretch{1.3}
\setlength\parindent{5ex}
\titlelabel{\thetitle.\quad}

\setitemize{topsep=0em, itemsep=0em}
\setenumerate{topsep=0em, itemsep=0em}

\begin{document}

\include{titlepage}

\tableofcontents
\newpage

\section{Цель работы}

Познакомить студентов с возможностями реализации более сложной обработки данных на стороне сервера с помощью хранимых процедур.

\section{Программа работы}

\begin{enumerate}
	\item Изучение возможностей языка \code{PL/pgSQL}.
	\item Создание двух хранимых процедур в соответствии с индивидуальным заданием, полученным у преподавателя.
	\item Выкладывание скрипта с созданными сущностями в репозиторий.
	\item Демонстрация результатов преподавателю.
\end{enumerate}

\section{Хранимые процедуры}

\subsection{Рекомендации по изменению количества номеров}

\paragraph{Задание:} Для заданного отеля и количества лет реализовать расчет рекомендаций по увеличению/уменьшению количества номеров каждого типа. Решение принимать на основе превышения/снижения значения ниже порогового.

Выберем следующий критерий эффективности заданного типа номера: если среднее число бронирований такого типа номеров выше заданного порога, то такой тип номера считается эффективным и можно рекомендовать увеличить число таких номеров. Неэффективные номера находятся по аналогии. Будем формировать итоговую процедуру поэтапно, демонстрируя промежуточные результаты для отеля с ID = 5:
\begin{enumerate}[leftmargin=0em]
	\item Определение количества бронирований номеров каждого типа для выбранного отеля:
	\sql{reservatrions_per_room_type}

	\item Определение количества номеров соответствующих каждому типу номеров:
	\sql{rooms_per_room_type}

	\item Формирование итоговой рекомендации по увеличению или уменьшению количества номеров данного типа в заданном отеле:
	\sql{recommend_rooms_optimization}
\end{enumerate}

В результирующем отчете выводится список типов номеров в отеле и соответствующие им рекомендации: необходимо ли увеличить количество номеров такого типа (\code{increase}) или уменьшить (\code{decrease}).

\paragraph{Измерение времени:} увеличим количество бронирований до 100 тысяч с помощью генератора и измерим время исполнения итогового запроса:

\begin{enumerate}
	\item execution time: 48 ms, fetching time: 22 ms
	\item execution time: 39 ms, fetching time: 5 ms
	\item execution time: 37 ms, fetching time: 9 ms
\end{enumerate}

Видно, что после первого запуска время выполнения запроса немного уменьшилось, что объясняется кэшированием результатов внутри базы данных.

\newpage

\subsection{Рекомендации на следующую поездку}

\paragraph{Задание:} На основе известных данных о госте (страны, длительности, отзывы и пр.) сформировать рекомендации на следующую поездку.

\noindent Будем рекомендовать отели пользователю по следующему критерию: 

\begin{itemize}
	\item найдем бронирования пользователя, в отзыве к которым он указал высокую оценку;
	\item найдем пользователей, которым понравился тот же номер в отеле;
	\item найдем у этих пользователей другие бронирования, в отзывах к которым был также указан высокий рейтинг, и будем рекомендовать пользователю отели,  в которых были оставлены эти отзывы.
\end{itemize}

\noindent Будем формировать итоговую процедуру поэтапно, демонстрируя промежуточные результаты для пользователя с ID = 3:
\begin{enumerate}[leftmargin=0em]
	\item Определение номеров, забронировав которые пользователь указал высокую оценку:
	\sql{liked_rooms}
	
	\item Определение пользователей, которые также ставили высокую оценку этим номерам: 
	\sql{similar_users}
	
	\item Формирование итоговой рекомендации на следующую поездку:
	\sql{recommend_hotel}
\end{enumerate}

\paragraph{Измерение времени:} увеличим количество бронирований до 100 тысяч с помощью генератора и измерим время исполнения итогового запроса:

\begin{enumerate}
	\item execution time: 91 ms, fetching time: 8 ms
	\item execution time: 68 ms, fetching time: 9 ms
	\item execution time: 60 ms, fetching time: 9 ms
\end{enumerate}

Видно, что после первого запуска время исполнения также несильно уменьшилось. 

\section{Выводы}

В процессе выполнения данной работы:

\begin{itemize}
	\item изучены возможности языка \code{PL/pgSQL};
	\item реализованы хранимые процедуры по заданию преподавателя: рекомендации по изменению количества номеров и рекомендации на следующую поездку.
\end{itemize}


\end{document}
