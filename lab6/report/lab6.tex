\documentclass[a4paper,14pt]{extarticle}

\usepackage[utf8x]{inputenc}
\usepackage[T1]{fontenc}
\usepackage[russian]{babel}
\usepackage{hyperref}
\usepackage{indentfirst}
\usepackage{here}
\usepackage{array}
\usepackage{graphicx}
\usepackage{caption}
\usepackage{subcaption}
\usepackage{chngcntr}
\usepackage{amsmath}
\usepackage{amssymb}
\usepackage[left=2cm,right=2cm,top=2cm,bottom=2cm,bindingoffset=0cm]{geometry}
\usepackage{multicol}
\usepackage{multirow}
\usepackage{titlesec}
\usepackage{listings}
\usepackage{color}
\usepackage{enumitem}
\usepackage{cmap}
\usepackage{underscore}

\definecolor{green}{rgb}{0,0.6,0}
\definecolor{gray}{rgb}{0.5,0.5,0.5}
\definecolor{purple}{rgb}{0.58,0,0.82}

\lstdefinelanguage{none}{}

\lstset{
	language={Java},
	inputpath={../generator/src/main/java/com/vaddya/hotelbooking},
	backgroundcolor=\color{white},
	commentstyle=\color{green},
	keywordstyle=\color{blue},
	numberstyle=\scriptsize\color{gray},
	stringstyle=\color{purple},
	basicstyle=\ttfamily\small,
	breakatwhitespace=false,
	breaklines=true,
	captionpos=b,
	keepspaces=true,
	numbers=left,
	numbersep=5pt,
	showspaces=false,
	showstringspaces=false,
	showtabs=false,
	tabsize=4,
	texcl=true,
	extendedchars=false,
	frame=single,
	morekeywords={IF, BIGSERIAL, SERIAL, TEXT, BIGINT, MONEY, BOOLEAN, REFERENCES}
}

\renewcommand{\le}{\ensuremath{\leqslant}}
\renewcommand{\leq}{\ensuremath{\leqslant}}
\renewcommand{\ge}{\ensuremath{\geqslant}}
\renewcommand{\geq}{\ensuremath{\geqslant}}
\renewcommand{\epsilon}{\ensuremath{\varepsilon}}
\renewcommand{\phi}{\ensuremath{\varphi}}
\renewcommand{\thefigure}{\arabic{figure}}
\newcommand{\code}[1]{\texttt{#1}}
\newcommand{\caret}{\^{}}

\titleformat*{\section}{\large\bfseries} 
\titleformat*{\subsection}{\normalsize\bfseries} 
\titleformat*{\subsubsection}{\normalsize\bfseries} 
\titleformat*{\paragraph}{\normalsize\bfseries} 
\titleformat*{\subparagraph}{\normalsize\bfseries} 

\counterwithin{figure}{section}
\counterwithin{equation}{section}
\counterwithin{table}{section}
\newcommand{\sign}[1][5cm]{\makebox[#1]{\hrulefill}}
\newcommand{\equipollence}{\quad\Leftrightarrow\quad}
\newcommand{\no}[1]{\overline{#1}}
\graphicspath{{../pics/}}
\captionsetup{justification=centering,margin=1cm}
\def\arraystretch{1.3}
\setlength\parindent{5ex}
\titlelabel{\thetitle.\quad}

\setitemize{topsep=0.5em, itemsep=0em}
\setenumerate{topsep=0.5em, itemsep=0em}

\begin{document}

\begin{titlepage}
\begin{center}
	Санкт-Петербургский Политехнический Университет Петра Великого\\[0.3cm]
	Институт компьютерных наук и технологий \\[0.3cm]
	Кафедра компьютерных систем и программных технологий\\[4cm]
	
	\textbf{ОТЧЕТ}\\ 
	\textbf{по лабораторной работе}\\[0.5cm]
	\textbf{<<Разработка структуры базы данных>>}\\[0.1cm]
	Базы данных\\[3.0cm]
\end{center}

\begin{flushright}
	\begin{minipage}{0.45\textwidth}
		\textbf{Работу выполнил студент}\\[3mm]
		группа 43501/3 \hfill Дьячков В.В.\\[5mm]
		\textbf{Работу принял преподаватель}\\[5mm]
		\sign[3cm] \hfill Мяснов А.В. \\[5mm]
	\end{minipage}
\end{flushright}

\vfill

\begin{center}
	Санкт-Петербург\\[0.3cm]
	\the\year
\end{center}
\end{titlepage}

\addtocounter{page}{1}

\tableofcontents
\newpage

\section{Цель работы}

Познакомить студентов с возможностями реализации более сложной обработки данных на стороне сервера с помощью хранимых процедур и триггеров.

\section{Программа работы}

\begin{enumerate}
	\item Создание двух триггеров: один триггер для автоматического заполнения ключевого поля, второй триггер для контроля целостности данных в подчиненной таблице при удалении/изменении записей в главной таблице.
	\item Создание триггера в соответствии с индивидуальным заданием, полученным у преподавателя.
	\item Создание триггера в соответствии с индивидуальным заданием, вызывающего хранимую процедуру.
	\item Выкладывание скрипта с созданными сущностями в GitLab.
	\item Демонстрация результатов преподавателю.
\end{enumerate}

\section{Хранимые процедуры}

\subsection{Автоматическое заполнение ключевого поля}

Создадим триггер для автоматического заполнения ID в таблице \code{room_type}. Внутри триггера, для наглядности, будем брать следующее значение последовательности \code{room_type_id_seq} и прибавлять 100.

\lstinputlisting[caption=\code{primary_key_trigger.sql}]{sql/primary_key_trigger.sql}

Видно, что триггер сработал и заполнил значение ключевого поля.

Контроль целостности данных в подчиненных таблицах обеспечивается ограничениями \code{constraints} внешних ключей.

\lstinputlisting[caption=\code{foreign_key_trigger.sql}]{sql/foreign_key_trigger.sql}

\vspace{-1em}
\subsection{Расчет стоимости бронирования}

\paragraph{Задание:} Расчет стоимости бронирования по данным стоимости номеров в разные моменты времени

С помощью функции \code{generate_series} сгенерируем набор дат, на которые забронирован номер (исключим последний, так как будем считать что день выезда не входит в стоимость бронирования).

Список сгенерированных дат соединим (\code{JOIN}) с таблицей цен по условию: ID комнат совпадают и сгенерированная дата попадает в интервал дат нужной цены. Тогда сумма цен (\code{price}) будет является рассчитанной ценой данного бронирования.

\lstinputlisting[caption=\code{calculate_price.sql}]{sql/calculate_price.sql}

Попробуем добавить номер, не указывая цену, причем для наглядности возьмем даты бронирования на границе период действия цен.

\lstinputlisting[caption=\code{calculate_price_example.sql}]{sql/calculate_price_example.sql}

Видно, что цена 10-дневного бронирования была рассчитана верно: 
\begin{displaymath}
7\text{ дней} \cdot \$2,666 + 3\text{ дня} \cdot \$6,319 = \$37,619,00.
\end{displaymath}

\subsection{Проверка доступности номера}

\paragraph{Задание:} При добавлении бронирования проверять доступность номера, в случае недоступности -- выбрасывать исключение.

Для проверки доступности номера на весь период бронирования выполним подзапрос для поиска конфликтующих бронирований (с тем же номером и указанным условием на даты). После этого, если список конфликтующих бронирований не пуст, то бросим исключение, содержащее поясняющее сообщение и список.

\lstinputlisting[caption=\code{reservation_validate_availability.sql}]{sql/reservation_validate_availability.sql}

Попробуем добавить бронирование, конфликтующее по датам с бронированием, созданным в предыдущем пункте. Для наглядности попробуем несколько разных комбинаций: новое начинается до предыдущего и кончается в середине предыдущего; новое начинается до предыдущего и кончается после предыдущего и т.д.

\lstinputlisting[caption=\code{reservation_validate_availability_example.sql}]{sql/reservation_validate_availability_example.sql}

Видно, что добавление конфликтующих бронирований было отклонено.

\vspace{-1em}
\section{Выводы}

В процессе выполнения данной работы:

\begin{itemize}
	\item изучены возможности языка \code{PL/pgSQL} ;
	\item создан триггер для автоматического заполнения ключевого поля;
	\item создан триггер для автоматического расчета стоимости бронирования;
	\item создан триггер для автоматической проверки доступности номера на указанные даты при добавлении нового бронирования.
\end{itemize}

\end{document}
